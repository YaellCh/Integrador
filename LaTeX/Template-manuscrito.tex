\documentclass{IEEEcsmag}

\usepackage[colorlinks,urlcolor=blue,linkcolor=blue,citecolor=blue]{hyperref}
\expandafter\def\expandafter\UrlBreaks\expandafter{\UrlBreaks\do\/\do\*\do\-\do\~\do\'\do\"\do\-}
\usepackage{upmath,color}

\usepackage[spanish]{babel}
%\usepackage[latin1]{inputenc}
\usepackage[utf8]{inputenc}  

\jvol{1}
\jnum{1}
\paper{1}
\jmonth{Noviembre}
\jname{ITICs letters}
\jtitle{Proyectos Integradores}
\pubyear{2023}

\newtheorem{theorem}{Theorem}
\newtheorem{lemma}{Lemma}



\setcounter{secnumdepth}{0}

\begin{document}

\sptitle{Proyecto Integrador de Primer Semestre}

\title{Software de resolución de problemas de Ingeniería }

\author{Yael Antonio Chavez Atanacio}
\affil{Instituto Tecnológico Superior del Occidente del Estado de Hidalgo, Mixquiahuala, Hgo., 42700, Mexico}

\author{Alejandro Cruz Martinez}
\affil{Instituto Tecnológico Superior del Occidente del Estado de Hidalgo, Mixquiahuala, Hgo., 42700, Mexico}

\author{Brayan Hernandez Martinez}
\affil{Instituto Tecnológico Superior del Occidente del Estado de Hidalgo, Mixquiahuala, Hgo., 42700, Mexico}

\author{Irving Maldonado Olguin}
\affil{Instituto Tecnológico Superior del Occidente del Estado de Hidalgo, Mixquiahuala, Hgo., 42700, Mexico}

\author{Gracie Hermione Gutierrez Vazquez}
\affil{Instituto Tecnológico Superior del Occidente del Estado de Hidalgo, Mixquiahuala, Hgo., 42700, Mexico}

%\author{Third Author III}
%\affil{Institute, City, (State), Postal Code, Country}

\markboth{ITSOEH/ITICS/PROYECTO INTEGRADOR PRIMER SEMESTRE}{THEME/FEATURE/DEPARTMENT}

\begin{abstract}
Un resumen (abstract) es un párrafo único que resume los aspectos importantes del manuscrito. A menudo indica si el manuscrito es un informe de un trabajo nuevo, una revisión o una descripción general, o una combinación de ambos. No cite referencias en el resumen. Este tipo de documento debe incluir contenido propiedad de los autores; es decir, no debe contener contenido de otras fuentes, ademas la redacción debe  estar dirigida a un tipo de lector técnico general. Este archivo se encuentra disponible en \href{https://github.com/fcuadrosgithub/integrador-primero.git}{https://github.com/fcuadrosgithub/integrador-primero.git}.
\end{abstract}

\maketitle
\chapteri{L}a introducción debe proporcionar información general (incluidas referencias relevantes) y debe indicar el propósito del manuscrito. En esta sección describa de manera clara y precisa el objetivo del proyecto integrador, la metodología que piensa usar y los resultados obtenidos de manera muy general. Dentro de esta sección puede citar trabajos relevantes de otros si lo cree necesario.

Esta sección debe dar un panorama muy general al lector de cual es el problema a resolver, que metodología utilizó para dar solución al problema y cuales fueron los resultados obtenidos. 

La redacción del manuscrito debe ser en tercera persona y queda estrictamente prohibido el uso de palabras coloquiales o Español informal. En lugar de esto utilice un lenguaje formal que el mayor numero de personas pueda entender.

\section{COPYRIGHT Y ACCESO ABIERTO}

Una vez que los autores entreguen este documento para su evaluación también seden los derechos del contenido de este manuscrito a la carrera de Ingeniería en Tecnológicas de la Información y Comunicaciones (ITICs) del Instituto Tecnológico Superior del Occidente del Estado de Hidalgo (ITSOEH). Esto conlleva que la carrera puede usar el contenido de este articulo para efectos de difusión del quehacer de los estudiantes de la carrera o en cualquier otra actividad que la carrera considere pertinente. Cabe mencionar que en ningún momento el orden o los nombres de los autores sera modificado de ninguna manera y siempre se les dará el crédito correspondiente. 
\section{PROBLEMAS}
A continuación se describen los problemas que el equipo deberá resolver.
\begin{enumerate}
\item Dados 2 puntos $A \mbox{ y } B$ con coordenadas $x_{1}, y_{1}$ y $x_{2}, y_{2}$  respectivamente. Regresar la ecuación de la recta y el ángulo interno $\alpha$ que se forma entre el eje horizontal y la recta. 
%Por ejemplo con los puntos $A(2, 1)$ y $B(-3, 2)$ la ecuación debe ser $y = -\frac{1}{5}x + \frac{7}{5}$. 
\item Dada una ecuación cuadratica regresar los valores de las raíces en caso de que estén sobre el conjunto de los números reales, en caso contrario indicar que la solución esta en el conjunto de los números complejos. 
\item Dada una circunferencia con centro en el punto $C$ con coordenadas $(x_{1}, y_{1})$ y radio $r$, evaluar si un punto $T$ con coordenadas $(x_{2}, y_{2})$ esta dentro del area de la circunferencia.
\item Dado un numero decimal entero positivo o negativo regresar su equivalente en binario.
\item Dado un numero binario de $n$ bits regresar su equivalente en decimal.
\item Dada una tabla de verdad de $n$ bits generar la expresión booleana que genere de manera fidedigna las salidas de esta tabla.
\end{enumerate}
%imprimir codigo
\lstnewenvironment{javaCode}[1][]
{\lstset{
    language=Java,
    basicstyle=\scriptsize\ttfamily,
    numbers=none, % Modificado: quitar los números de línea
    keywordstyle=\color{blue},
    commentstyle=\color{gray},
    stringstyle=\color{purple},
    breaklines=true,
    breakatwhitespace=true,
    tabsize=4,
    showspaces=false,
    showstringspaces=false,
    frame=single,
    captionpos=b,
    floatplacement=!h,
    #1
}}
{}

\section{Sección Problema 1}
\item Dados 2 puntos $A \mbox{ y } B$ con coordenadas $x_{1}, y_{1}$ y $x_{2}, y_{2}$  respectivamente. Regresar la ecuación de la recta y el ángulo interno $\alpha$ que se forma entre el eje horizontal y la recta.

\section{Introducción}
El problema abordado en este proyecto se centra en la determinación de la ecuación de la recta y el ángulo interno $\alpha$ 
formado entre dicha recta y el eje horizontal. Dados dos puntos, $A$ y $B$, con coordenadas $(x_1, y_1)$ y $(x_2, y_2)$ respectivamente, 
el objetivo es desarrollar una solución matemática que permita calcular la pendiente de la recta, la intersección en el eje $Y$ y el ángulo $\alpha$.

\section{Descripción del problema}
En el contexto de geometría analítica, la tarea es obtener la ecuación de la recta que pasa por los puntos $A$ y $B$, 
así como determinar el ángulo interno $\alpha$ que se forma entre la recta y el eje horizontal. 
La solución implica el uso de conceptos trigonométricos y algebraicos para abordar este problema geométrico de manera precisa y eficiente.

\section{Diseño de la solución}
La solución propuesta emplea tres funciones matemáticas clave. Primero, se calcula la inclinación de la recta mediante la función \texttt{inclinación}, 
luego se determina la intersección en el eje $Y$ con la función \texttt{intersección}, y finalmente se calcula el ángulo interno $\alpha$ con la función \texttt{ángulo}. 
Estas funciones trabajan en conjunto para proporcionar la ecuación de la recta y el ángulo deseado.

\section{Desarrollo de la solución}
En la ecuación de la recta, si dos puntos distintos $P(x_{1}, y_{1})$ y $Q(x_{2}, y_{2})$ se ubican en la curva $y=f(x)$, 
la pendiente de la recta secante que une los dos puntos es:

\begin{equation}
    m_{sec}=\frac{y_{2} - y_{1}}{x_{2} - x_{1}} = \frac{f_{(x2)} - f_{(x1)} }{x_{2} - x_{1}}/
    \label{eqn:rectaPendiente}
\end{equation}

La forma punto-pendiente de la ecuación de la recta, con una coordenada especifica en el plano cartesiano se define como:
\begin{equation}
    b = y_{1} - m * x_{1}
\end{equation}

\begin{figure}[h!]
    \centering
    \includegraphics[width = 6 cm]{imagen/GraficaEcuacionRecta.png}
    \caption{Gráfica de la ecuación de la recta}
    \label{fig:GraficaEcuacionRecta}
\end{figure}

Utilizando este método, puedes encontrar la ecuación de la recta a partir de dos puntos. Recuerda que si los dos puntos son idénticos la recta será una linea vertical \cite{rectaPendiente}

El algoritmo de solución para encontrar la ecuación de la recta pendiente  (ec. \ref{eqn:rectaPendiente}) comienza solicitando al usuario dos puntos $P(x_{1}, y_{1})$ y $Q(x_{2}, y_{2})$.

\begin{javaCode}

Scanner puntos = new Scanner (System.in);
        
    //Solicitar puntos para la ecuacion de la recta  
    System.out.println("""
                        Ingresa las coordenadas del punto 1.
                        seperadas por una coma (x,y):
                           """);
    
    String[] punto1 =puntos.nextLine().split(",");
        
    System.out.println("""
                        Ingresa las coordenadas del punto 1.
                        seperadas por una coma (x,y):
                        """);
    
    String[] punto2 =puntos.nextLine().split(",");
        
    //cerrar el escaneo
    puntos.close();
        
\end{javaCode}

Posteriormente se convierten los valores de los puntos $P(x_{1}, y_{1})$ y $Q(x_{2}, y_{2})$. en valores enteros para ser utilizados  en la ecuación de la recta (ec. \ref{eqn:rectaPendiente}).

\begin{javaCode}
    //Asignar valor de coordenadas a x,y para dos puntos
    int x1= Integer.parseInt(punto1[0].trim());
    int y1= Integer.parseInt(punto1[1].trim());
       
    int x2= Integer.parseInt(punto2[0].trim());
    int y2= Integer.parseInt(punto2[1].trim());
\end{javaCode}

Una vez normalizado los datos, se calcula la inclinación de la recta y calcular la intersección de la recta.

\begin{javaCode}
    //Calculo para la inclinacion de la recta  
    Double m = (double)(y2 - y1)/(x2 - x1);
       
    //Calcular Interseccion de la recta
    Double b= y1 -(m * x1);
\end{javaCode}

Por ultimo, se imprime la ecuación de la recta:
\begin{javaCode}
    //Imprimir el resultado 
        System.out.println("Ecuacion de la recta: \n" +
                            m + " x + " + b +" y ");
\end{javaCode}

\section{Depuración y pruebas}
\begin{tabular}{|c|c|c|c|c|c|}
    \hline
    \textbf{Número de Prueba} & \textbf{Datos Ingresados} & \textbf{\(\boldsymbol{x_1}\)} & \textbf{\(\boldsymbol{y_1}\)} & \textbf{\(\boldsymbol{x_2}\)} & \textbf{\(\boldsymbol{y_2}\)} \\
    \hline
    1 & (2, 3), (4, 7) & 2 & 3 & 4 & 7 \\
    \hline
    2 & (-1, 0), (3, 4) & -1 & 0 & 3 & 4 \\
    \hline
    3 & (0, 0), (0, 5) & 0 & 0 & 0 & 5 \\
    \hline
    \end{tabular}
    
    \vspace{0.5cm}
    
    \begin{tabular}{|c|c|c|}
    \hline
    \textbf{Número de Prueba} & \textbf{\(m\)} & \textbf{\(b\)} \\
    \hline
    1 & 2.0 & -1.0 \\
    \hline
    2 & 1.0 & 1.0 \\
    \hline
    3 & \text{--} & \text{--} \\
    \hline
    \end{tabular}
    
    \vspace{0.5cm}
    
    \begin{tabular}{|c|c|}
    \hline
    \textbf{Número de Prueba} & \textbf{\(\alpha\)} \\
    \hline
    1 & \(63.43^\circ\) \\
    \hline
    2 & \(63.43^\circ\) \\
    \hline
    3 & \(90^\circ\) \\
    \hline
    \end{tabular}

\newpage
Contenido del primer problema...
\newpage


\section{Sección Problema 2}
Contenido del segundo problema para Irving...
\newpage
Contenido del segundo problema...
\newpage


\sección{Problema 3:Cálculo de la distancia entre dos puntos en el área de una circunferencia}
Contenido del tercer problema para Alejandro...
\newpage 
\begin{abstract}
    
    El reporte del problema indica como calcular la distancia de dos puntos y su ubicación ya que se puede encontrar dentro o fuera de una circunferencia además también nos da una posible solución a través de un programa 
\end{abstract}


\section{Introducción}
El cálculo de la distancia entre dos puntos y determinar si se encuentran dentro de una circunferencia es una tarea común en programación . Para esto podemos utilizar un programa en java que realice el cálculo de la distancia y su ubicación dado por dos puntos ingresados por el usuario 

\section{Desarrollo}
La formula de la distancia entre dos puntos $p = (x_1, y_1)$ y $q = (x_2, y_2)$, la distancia entre los dos puntos se calcula utilizando la fórmula de la distancia:
\text{distancia} = $\sqrt{{(x_2 - x_1)^2 + (y_2 - y_1)^2}}$,
Esta se utiliza porque proporciona una medida precisa de la distancia entre dos puntos en un plano. Además, al elevar al cuadrado las coordenadas, se eliminan los signos negativos y se preserva la información sobre la magnitud  de las diferencias. Esto puede ser útil en ciertos contextos donde solo se necesita comparar distancias y no se requiere la distancia exacta en sí, ejemplo:en telecomunicaciones para saber si una antena de recepción de red se encuentra en su zona o la distancia en la que se encuentra.
también se utilizo una condicional (if) para saber si se encuentra dentro o fuera de la circunferencia dada por el radio ingresado por el usuario.


\begin{figure}[h!]
    \centering
    \includegraphics[width = 6 cm]{latex-imagenes/imagen7.png}
    \caption{Gráfica de la distancia de dos puntos}
    \label{fig:Grafica de la distancia de dos puntos }
\end{figure}

\section{Diseño de solución}
\begin{itemize}
    \item se solicita al usuario ingresar los valores (x1,y1) radio y (x2,y2)
    \item procedemos a calcular la distancia con la formula antes mencionada
    \item comprobar que el radio sea positivo y sino pasarlo a positivo
    \item calcular si el punto se encuentra dentro de la circunferencia
    
\end{itemize}

\section{Diagrama de flujo}
A continuación veremos el diagrama de flujo que fue la base para el desarrollo del programa
\begin{figure}[h!]
    \centering
    \includegraphics[width = 9 cm]{latex-imagenes/Imagen2.png}
    \caption{se muestra el diagrama que fue base para diseñar el programa}
    \label{fig:diagrama de flujo}
\end{figure}
\section{Programa en Java}
A continuación, se muestra el código en Java para calcular la distancia entre dos puntos y verificar si se encuentran dentro de una circunferencia:


   \begin{javaCode}
       
    public static void main(String[] args) {
        
         Scanner dato = new Scanner(System.in);
        \end{javaCode}
        solicitamos al usuario las cordenadas el punto C (x1,y1)
       \begin{javaCode}
        System.out.print("Ingrese las coordenada del centro de la circunferencia primero x1: ");
        int x1 = dato.nextInt();
        System.out.print("Ingrese las coordenada del centro de la circunferencia despues y1: ");
        int y1 = dato.nextInt();
         \end{javaCode}
         solicitamos el radio de la circunferencia
       
        \begin{javaCode}
        System.out.println("Ingrese el radio de la circunferencia: ");
        int radio = dato.nextInt();
         \end{javaCode}
         
        solicitamos al usuario las coordenadas del punto T (x2,y2)
       \begin{javaCode}
        System.out.print("Ingrese las coordenada del punto T X2: ");
        int x2 = dato.nextInt();
        System.out.print("Ingrese las coordenada del punto T Y2: ");
        int y2 = dato.nextInt();
        \end{javaCode}
         Cálculo de la distancia entre el punto C y el punto T
        \begin{javaCode}
        double distancia = Math.sqrt((x2 - x1) * (x2 - x1) + (y2 - y1) * (y2 - y1));
        \end{javaCode}
         imprimimos la distancia
          \begin{javaCode}
         System.out.println("la distancia es de "+distancia);
          \end{javaCode}
       
        si el radio es negativo lo pasamos a positivo
         \begin{javaCode}
        int radio2=radio*-1;
\end{javaCode}
         Si la distancia es menor o igual al radio, el punto T está dentro de la circunferencia     
       
        \begin{javaCode}
                    if(radio<0){
            
        
        int radio2=radio*-1;
        if (distancia <= radio2) {
            System.out.println("El punto T está dentro de la circunferencia");
        } else {
            System.out.println("El punto T no está dentro de la circunferencia");
        }
        }
        
       if(radio>0){
           
       
            if (distancia <= radio) {
            System.out.println("El punto T está dentro de la circunferencia");
        } else {
            System.out.println("El punto T no está dentro de la circunferencia");
        }
    }
    
    
    }
    
}



  
 
    \end{javaCode}
        \section{tabla de corridas}
 A continuación se muestra una tabla de corridas o pruebas que se realizaron con el programa en java:



   \begin{figure}[h!]
    \centering
    \includegraphics[width = 10 cm]{latex-imagenes/Imagen3.png}
    \caption{Tabla de pruebas}
    \label{fig:Grafica de la distancia de dos puntos }
\end{figure}




\section{Conclusión}
El programa se encarga de encontrar la distancia entre dos puntos en el área de una circunferencia, con ella encuentra la distancia entre dos puntos, luego es importante solicitar al usuario el radio, se transforma en positivo el radio en caso de ser negativo, con ello podremos ver si el punto t se encuentra dentro del área 
   
\begin{thebibliography}{00}
    \bibitem{rectaPendiente}
    Fórmula de la distancia (artículo). (s/f). Khan Academy. Recuperado el 21 de noviembre de 2023, de https://es.khanacademy.org/math/geometry/hs-geo-analytic-geometry/hs-geo-distance-and-midpoints/a/distance-formula.
    
    \bibitem[calculo de la distancia]
    Fórmula de la distancia. (2011, February 20).https://es.khanacademy.org/math/geometry/hs-geo-analytic-geometry/hs-geo-distance-and-midpoints/v/distance-formula}
    {https://es.khanacademy.org/math/geometry/hs-geo-analytic-geometry/hs-geo-distance-and-midpoints/v/distance-formula}

\end{thebibliography}












\newpage 


\section{Sección Problema 4}
Contenido del cuarto problema para Yael ...
\newpage 
Contenido del cuarto problema...
\newpage 


\section{Sección Problema 5}
Contenido del quinto problema para Brian...
\newpage 
    
\begin{document}

\title{Conversión de números binarios a decimales con programa en Java}
\author{Brayan Hernandez Martinez / Mixquiahuala de Juarez / 230110578@itsoeh.edu.mx}
\maketitle

%Resumen, se escribe al final y contiene máximo 250 palabras
\begin{abstract}
El reporte analiza el concepto de como se hace la conversion de un numero binario a numero decimal, dando la informacion necesaria para comprender la funcion del programa, los pasos que lleva acabo y la explicacion se lleva acabo. 
\end{abstract}

\section{Introducción}
En el campo de la computación, es común la necesidad de convertir números binarios a su equivalente en decimal. Para facilitar esta conversión, se puede utilizar un programa en Java que realice el cálculo de forma automática. En este artículo, presentamos un programa en Java que toma un número binario ingresado por el usuario y lo convierte a su equivalente en decimal.

\section{Diseño de solucion }

\begin{itemize}
  \item Se solicita al usuario ingresar un numero binario de n bits.
  \item El numero solo se leera si son 0's, 1's 
  \item Se caculara el numero dando su equivalente e decimal
  \item Se lanzara el resulatdo
\end{itemize}

\section{Programa en Java}
A continuación se muestra el código del programa en Java para la conversión de números binarios a decimales:

\begin{lstlisting}[style=javaStyle]
import java.util.Scanner;
import java.math.BigInteger;

public class NewClass {
    public static void main(String[] args) {
        Scanner bin = new Scanner(System.in);
        boolean continuar = true;
\end{lstlisting}
En este programa se uso un bluce el cual nos va a permitir volver a usar este programa las veces que sean necesarias.

\section{Pasos}
// Se solicita al usuario ingresar un numero binario
\begin{lstlisting}[style=javaStyle]
// Bucle que permite al usuario ingresar números binarios hasta que ingrese 'x'

        while (continuar) {
        
            System.out.print("Ingrese un número en binario (o 'x' para terminar ): ");
            
            String entrada = bin.nextLine();
\end{lstlisting}

// Para el usuario al no requerir mas del programa al seleccionar x el programa dejara de repetir el ciclo y finalizara.

\begin{lstlisting}[style=javaStyle]
                 // Si se ingresa 'x', el bucle se detiene
            if (entrada.equalsIgnoreCase("x")) {
                continuar = false;
                continue;
            }
\end{lstlisting}

// En esta línea, se declara una variable llamada decimal del tipo BigInteger. Se utiliza para almacenar el valor decimal resultante después de convertir el número binario ingresado. La función convertirBinarioADecimal se llama con el argumento entrada, que es el número binario ingresado por el usuario.

\begin{lstlisting}[style=javaStyle]
// Llama al método convertirBinarioADecimal para convertir el número binario ingresado a decimal
            BigInteger decimal = convertirBinarioADecimal(entrada);
            
\end{lstlisting}

// Esta línea imprime en la consola el mensaje "El número en decimal es: " seguido del valor almacenado en la variable decimal. Es decir, muestra el resultado de la conversión en formato decimal.

\begin{lstlisting}[style=javaStyle]
System.out.println("El número en decimal es: " + decimal);

\end{lstlisting}

// Estas líneas definen el método convertirBinarioADecimal, que toma un argumento de tipo String llamado binario. El método crea un nuevo objeto BigInteger utilizando el constructor que toma dos argumentos: el número binario (binario) y la base (2) que se utiliza para interpretar el número binario y convertirlo a decimal. 

\begin{lstlisting}[style=javaStyle]
lic static BigInteger convertirBinarioADecimal(String binario) {
        return new BigInteger(binario, 2);

\end{lstlisting}
\section{Solución}
Para convertir un número binario a decimal, se utiliza la siguiente ecuación:

\[
\text{{decimal}} = \sum_{i=0}^{n-1} \left( \text{{binarioString}}[i] \times 2^{n-1-i} \right)
\]

\text{{binarioString}} Donde es el número binario ingresado por el usuario, $n$ es la longitud del número binario y $i$ es la posición del dígito en el número binario. En cada iteración del bucle, se verifica si el dígito en la posición $i$ es igual a '1'. Si es así, se suma el valor correspondiente a {{decimal}} utilizando la base actual, que se incrementa en cada iteración multiplicándola por 2.

\section{Codigo en general }
\begin{lstlisting}[style=javaStyle]
import java.util.Scanner;
import java.math.BigInteger;

public class NewClassimport java.math.BigInteger;
    public static void main(String[] args) {
        Scanner bin = new Scanner(System.in);
        boolean continuar = true;
       
        // Bucle que permite al usuario ingresar números binarios hasta que ingrese 'x'
        while (continuar) {
            System.out.print("Ingrese un número en binario (o 'x' para terminar ): ");
            String entrada = bin.nextLine();

             // Si se ingresa 'x', el bucle se detiene
            if (entrada.equalsIgnoreCase("x")) {
                continuar = false;
                continue;
            }

            // Llama al método convertirBinarioADecimal para convertir el número binario ingresado a decimal
            BigInteger decimal = convertirBinarioADecimal(entrada);
            System.out.println("El número en decimal es: " + decimal);
        }
    }

    public static BigInteger convertirBinarioADecimal(String binario) {
        return new BigInteger(binario, 2);
    }
}
\end{lstlisting}

\section{Diagrama de flujo}


\tikzstyle{startstop} = [rectangle, rounded corners, minimum width=3cm, minimum height=1cm,text centered, draw=black, fill=red!30]
\tikzstyle{io} = [trapezium, trapezium left angle=70, trapezium right angle=110, minimum width=3cm, minimum height=1cm, text centered, draw=black, fill=blue!30]
\tikzstyle{process} = [rectangle, minimum width=3cm, minimum height=1cm, text centered, draw=black, fill=orange!30]
\tikzstyle{decision} = [diamond, minimum width=3cm, minimum height=1cm, text centered, draw=black, fill=green!30]
\tikzstyle{arrow} = [thick,->,>=stealth]

\begin{tikzpicture}[node distance=2cm]

\node (start) [startstop] {Comienzo};
\node (input) [io, below of=start] {Ingresar número binario};
\node (decision) [decision, below of=input, yshift=-0.5cm] {¿Es 'x'?};
\node (stop) [startstop, below of=decision, yshift=-0.5cm] {Terminar};
\node (convert) [process, right of=decision, xshift=2cm] {Convertir a decimal};
\node (output) [io, below of=convert] {Mostrar número decimal};

\draw [arrow] (start) -- (input);
\draw [arrow] (input) -- (decision);
\draw [arrow] (decision) -- node[anchor=south] {sí} (stop);
\draw [arrow] (decision) -- node[anchor=south] {no} (convert);
\draw [arrow] (convert) -- (output);
\draw [arrow] (output) -- +(0,-1.5cm) -- +(-4.5cm,-1.5cm) -- +(0,-0.5cm) -- (input);

\end{tikzpicture}
\begin{figure}[h!]
\centering
    \includegraphics[width = 6 cm]{}
    \caption{Diagrama de flujo}
\end{figure}

\section{Tabla de corridas}
\begin{table}[ht]
  \centering
  \large
  \begin{tabularx}{\textwidth}{|X|X|X|}
    \hline
    \textbf{Num. De corridas} & \textbf{Ejemplo} & \textbf{Resultado} \\
    \hline
    Num 1 & 1100 & 12 \\
    Num 2 & 10111 & 23 \\
    Num 3 & 1001111 & 79 \\
    Num 4 & 1000101 & 69 \\
    Num 5 & 101010100 & 340 \\
    \hline
  \end{tabularx}
\end{table}


\section{Conclusiones}
En este artículo, se presentó un programa en Java para convertir números binarios a su equivalente en decimal. El programa utiliza un enfoque directo y eficiente, aprovechando la clase BigInteger de Java. Al proporcionar un número binario como entrada, el programa realiza los cálculos necesarios y muestra el resultado en formato decimal.

La conversión de números binarios a decimales es un proceso fundamental en el campo de la computación y tiene aplicaciones en diversos escenarios. El programa presentado ofrece una solución práctica y sencilla para llevar a cabo esta conversión.
\end{document}

\newpage 


\section{Sección Problema 6}
Contenido del sexto problema para Gracie...
\newpage 
Contenido del sexto problema...
\newpage 



\end{document}

