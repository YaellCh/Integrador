\documentclass{IEEEcsmag}

\usepackage[colorlinks,urlcolor=blue,linkcolor=blue,citecolor=blue]{hyperref}
\expandafter\def\expandafter\UrlBreaks\expandafter{\UrlBreaks\do\/\do\*\do\-\do\~\do\'\do\"\do\-}
\usepackage{upmath,color}

\usepackage[spanish]{babel}
%\usepackage[latin1]{inputenc}
\usepackage[utf8]{inputenc}  

\jvol{1}
\jnum{1}
\paper{1}
\jmonth{Noviembre}
\jname{ITICs letters}
\jtitle{Proyectos Integradores}
\pubyear{2023}

\newtheorem{theorem}{Theorem}
\newtheorem{lemma}{Lemma}



\setcounter{secnumdepth}{0}

\begin{document}

\sptitle{Proyecto Integrador de Primer Semestre}

\title{Software de resolución de problemas de Ingeniería }

\author{Yael Antonio Chavez Atanacio}
\affil{Instituto Tecnológico Superior del Occidente del Estado de Hidalgo, Mixquiahuala, Hgo., 42700, Mexico}

\author{Alejandro Cruz Martinez}
\affil{Instituto Tecnológico Superior del Occidente del Estado de Hidalgo, Mixquiahuala, Hgo., 42700, Mexico}

\author{Brayan Hernandez Martinez}
\affil{Instituto Tecnológico Superior del Occidente del Estado de Hidalgo, Mixquiahuala, Hgo., 42700, Mexico}

\author{Irving Maldonado Olguin}
\affil{Instituto Tecnológico Superior del Occidente del Estado de Hidalgo, Mixquiahuala, Hgo., 42700, Mexico}

\author{Gracie Hermione Gutierrez Vazquez}
\affil{Instituto Tecnológico Superior del Occidente del Estado de Hidalgo, Mixquiahuala, Hgo., 42700, Mexico}

%\author{Third Author III}
%\affil{Institute, City, (State), Postal Code, Country}

\markboth{ITSOEH/ITICS/PROYECTO INTEGRADOR PRIMER SEMESTRE}{THEME/FEATURE/DEPARTMENT}

\begin{abstract}
Un resumen (abstract) es un párrafo único que resume los aspectos importantes del manuscrito. A menudo indica si el manuscrito es un informe de un trabajo nuevo, una revisión o una descripción general, o una combinación de ambos. No cite referencias en el resumen. Este tipo de documento debe incluir contenido propiedad de los autores; es decir, no debe contener contenido de otras fuentes, ademas la redacción debe  estar dirigida a un tipo de lector técnico general. Este archivo se encuentra disponible en \href{https://github.com/fcuadrosgithub/integrador-primero.git}{https://github.com/fcuadrosgithub/integrador-primero.git}.
\end{abstract}

\maketitle
\chapteri{L}a introducción debe proporcionar información general (incluidas referencias relevantes) y debe indicar el propósito del manuscrito. En esta sección describa de manera clara y precisa el objetivo del proyecto integrador, la metodología que piensa usar y los resultados obtenidos de manera muy general. Dentro de esta sección puede citar trabajos relevantes de otros si lo cree necesario.

Esta sección debe dar un panorama muy general al lector de cual es el problema a resolver, que metodología utilizó para dar solución al problema y cuales fueron los resultados obtenidos. 

La redacción del manuscrito debe ser en tercera persona y queda estrictamente prohibido el uso de palabras coloquiales o Español informal. En lugar de esto utilice un lenguaje formal que el mayor numero de personas pueda entender.

\section{COPYRIGHT Y ACCESO ABIERTO}

Una vez que los autores entreguen este documento para su evaluación también seden los derechos del contenido de este manuscrito a la carrera de Ingeniería en Tecnológicas de la Información y Comunicaciones (ITICs) del Instituto Tecnológico Superior del Occidente del Estado de Hidalgo (ITSOEH). Esto conlleva que la carrera puede usar el contenido de este articulo para efectos de difusión del quehacer de los estudiantes de la carrera o en cualquier otra actividad que la carrera considere pertinente. Cabe mencionar que en ningún momento el orden o los nombres de los autores sera modificado de ninguna manera y siempre se les dará el crédito correspondiente. 
\section{PROBLEMAS}
A continuación se describen los problemas que el equipo deberá resolver.
\begin{enumerate}
\item Dados 2 puntos $A \mbox{ y } B$ con coordenadas $x_{1}, y_{1}$ y $x_{2}, y_{2}$  respectivamente. Regresar la ecuación de la recta y el ángulo interno $\alpha$ que se forma entre el eje horizontal y la recta. 
%Por ejemplo con los puntos $A(2, 1)$ y $B(-3, 2)$ la ecuación debe ser $y = -\frac{1}{5}x + \frac{7}{5}$. 
\item Dada una ecuación cuadratica regresar los valores de las raíces en caso de que estén sobre el conjunto de los números reales, en caso contrario indicar que la solución esta en el conjunto de los números complejos. 
\item Dada una circunferencia con centro en el punto $C$ con coordenadas $(x_{1}, y_{1})$ y radio $r$, evaluar si un punto $T$ con coordenadas $(x_{2}, y_{2})$ esta dentro del area de la circunferencia.
\item Dado un numero decimal entero positivo o negativo regresar su equivalente en binario.
\item Dado un numero binario de $n$ bits regresar su equivalente en decimal.
\item Dada una tabla de verdad de $n$ bits generar la expresión booleana que genere de manera fidedigna las salidas de esta tabla.
\end{enumerate}
%imprimir codigo
\lstnewenvironment{javaCode}[1][]
{\lstset{
    language=Java,
    basicstyle=\scriptsize\ttfamily,
    numbers=none, % Modificado: quitar los números de línea
    keywordstyle=\color{blue},
    commentstyle=\color{gray},
    stringstyle=\color{purple},
    breaklines=true,
    breakatwhitespace=true,
    tabsize=4,
    showspaces=false,
    showstringspaces=false,
    frame=single,
    captionpos=b,
    floatplacement=!h,
    #1
}}
{}

\section{Sección Problema 1}
\item Dados 2 puntos $A \mbox{ y } B$ con coordenadas $x_{1}, y_{1}$ y $x_{2}, y_{2}$  respectivamente. Regresar la ecuación de la recta y el ángulo interno $\alpha$ que se forma entre el eje horizontal y la recta.

\section{Introducción}
El problema abordado en este proyecto se centra en la determinación de la ecuación de la recta y el ángulo interno $\alpha$ 
formado entre dicha recta y el eje horizontal. Dados dos puntos, $A$ y $B$, con coordenadas $(x_1, y_1)$ y $(x_2, y_2)$ respectivamente, 
el objetivo es desarrollar una solución matemática que permita calcular la pendiente de la recta, la intersección en el eje $Y$ y el ángulo $\alpha$.

\section{Descripción del problema}
En el contexto de geometría analítica, la tarea es obtener la ecuación de la recta que pasa por los puntos $A$ y $B$, 
así como determinar el ángulo interno $\alpha$ que se forma entre la recta y el eje horizontal. 
La solución implica el uso de conceptos trigonométricos y algebraicos para abordar este problema geométrico de manera precisa y eficiente.

\section{Diseño de la solución}
La solución propuesta emplea tres funciones matemáticas clave. Primero, se calcula la inclinación de la recta mediante la función \texttt{inclinación}, 
luego se determina la intersección en el eje $Y$ con la función \texttt{intersección}, y finalmente se calcula el ángulo interno $\alpha$ con la función \texttt{ángulo}. 
Estas funciones trabajan en conjunto para proporcionar la ecuación de la recta y el ángulo deseado.

\section{Desarrollo de la solución}
En la ecuación de la recta, si dos puntos distintos $P(x_{1}, y_{1})$ y $Q(x_{2}, y_{2})$ se ubican en la curva $y=f(x)$, 
la pendiente de la recta secante que une los dos puntos es:

\begin{equation}
    m_{sec}=\frac{y_{2} - y_{1}}{x_{2} - x_{1}} = \frac{f_{(x2)} - f_{(x1)} }{x_{2} - x_{1}}/
    \label{eqn:rectaPendiente}
\end{equation}

La forma punto-pendiente de la ecuación de la recta, con una coordenada especifica en el plano cartesiano se define como:
\begin{equation}
    b = y_{1} - m * x_{1}
\end{equation}

\begin{figure}[h!]
    \centering
    \includegraphics[width = 6 cm]{imagen/GraficaEcuacionRecta.png}
    \caption{Gráfica de la ecuación de la recta}
    \label{fig:GraficaEcuacionRecta}
\end{figure}

Utilizando este método, puedes encontrar la ecuación de la recta a partir de dos puntos. Recuerda que si los dos puntos son idénticos la recta será una linea vertical \cite{rectaPendiente}

El algoritmo de solución para encontrar la ecuación de la recta pendiente  (ec. \ref{eqn:rectaPendiente}) comienza solicitando al usuario dos puntos $P(x_{1}, y_{1})$ y $Q(x_{2}, y_{2})$.

\begin{javaCode}

Scanner puntos = new Scanner (System.in);
        
    //Solicitar puntos para la ecuacion de la recta  
    System.out.println("""
                        Ingresa las coordenadas del punto 1.
                        seperadas por una coma (x,y):
                           """);
    
    String[] punto1 =puntos.nextLine().split(",");
        
    System.out.println("""
                        Ingresa las coordenadas del punto 1.
                        seperadas por una coma (x,y):
                        """);
    
    String[] punto2 =puntos.nextLine().split(",");
        
    //cerrar el escaneo
    puntos.close();
        
\end{javaCode}

Posteriormente se convierten los valores de los puntos $P(x_{1}, y_{1})$ y $Q(x_{2}, y_{2})$. en valores enteros para ser utilizados  en la ecuación de la recta (ec. \ref{eqn:rectaPendiente}).

\begin{javaCode}
    //Asignar valor de coordenadas a x,y para dos puntos
    int x1= Integer.parseInt(punto1[0].trim());
    int y1= Integer.parseInt(punto1[1].trim());
       
    int x2= Integer.parseInt(punto2[0].trim());
    int y2= Integer.parseInt(punto2[1].trim());
\end{javaCode}

Una vez normalizado los datos, se calcula la inclinación de la recta y calcular la intersección de la recta.

\begin{javaCode}
    //Calculo para la inclinacion de la recta  
    Double m = (double)(y2 - y1)/(x2 - x1);
       
    //Calcular Interseccion de la recta
    Double b= y1 -(m * x1);
\end{javaCode}

Por ultimo, se imprime la ecuación de la recta:
\begin{javaCode}
    //Imprimir el resultado 
        System.out.println("Ecuacion de la recta: \n" +
                            m + " x + " + b +" y ");
\end{javaCode}

\section{Depuración y pruebas}
\begin{tabular}{|c|c|c|c|c|c|}
    \hline
    \textbf{Número de Prueba} & \textbf{Datos Ingresados} & \textbf{\(\boldsymbol{x_1}\)} & \textbf{\(\boldsymbol{y_1}\)} & \textbf{\(\boldsymbol{x_2}\)} & \textbf{\(\boldsymbol{y_2}\)} \\
    \hline
    1 & (2, 3), (4, 7) & 2 & 3 & 4 & 7 \\
    \hline
    2 & (-1, 0), (3, 4) & -1 & 0 & 3 & 4 \\
    \hline
    3 & (0, 0), (0, 5) & 0 & 0 & 0 & 5 \\
    \hline
    \end{tabular}
    
    \vspace{0.5cm}
    
    \begin{tabular}{|c|c|c|}
    \hline
    \textbf{Número de Prueba} & \textbf{\(m\)} & \textbf{\(b\)} \\
    \hline
    1 & 2.0 & -1.0 \\
    \hline
    2 & 1.0 & 1.0 \\
    \hline
    3 & \text{--} & \text{--} \\
    \hline
    \end{tabular}
    
    \vspace{0.5cm}
    
    \begin{tabular}{|c|c|}
    \hline
    \textbf{Número de Prueba} & \textbf{\(\alpha\)} \\
    \hline
    1 & \(63.43^\circ\) \\
    \hline
    2 & \(63.43^\circ\) \\
    \hline
    3 & \(90^\circ\) \\
    \hline
    \end{tabular}

\newpage
Contenido del primer problema...
\newpage


\section{Sección Problema 2}
Contenido del segundo problema para Irving...
\newpage
Contenido del segundo problema...
\newpage


\section{Sección Problema 3}
Contenido del tercer problema para Alejandro...
\newpage 
Contenido del tercer problema...
\newpage 


\section{Sección Problema 4}
Contenido del cuarto problema para Yael ...
\newpage 
Contenido del cuarto problema...
\newpage 


\section{Sección Problema 5}
Contenido del quinto problema para Brian...
\newpage 
Contenido del quinto problema...
\newpage 


\section{Sección Problema 6}
Contenido del sexto problema para Gracie...
\newpage 
Contenido del sexto problema...
\newpage 



\end{document}

