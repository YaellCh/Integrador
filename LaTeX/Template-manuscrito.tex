\documentclass{IEEEcsmag}

\usepackage[colorlinks,urlcolor=blue,linkcolor=blue,citecolor=blue]{hyperref}
\expandafter\def\expandafter\UrlBreaks\expandafter{\UrlBreaks\do\/\do\*\do\-\do\~\do\'\do\"\do\-}
\usepackage{upmath,color}

\usepackage[spanish]{babel}
%\usepackage[latin1]{inputenc}
\usepackage[utf8]{inputenc}  

\jvol{1}
\jnum{1}
\paper{1}
\jmonth{Noviembre}
\jname{ITICs letters}
\jtitle{Proyectos Integradores}
\pubyear{2023}

\newtheorem{theorem}{Theorem}
\newtheorem{lemma}{Lemma}



\setcounter{secnumdepth}{0}

\begin{document}

\sptitle{Proyecto Integrador de Primer Semestre}

\title{Software de resolución de problemas de Ingeniería }

\author{Cuadros Romero Francisco Javier}
\affil{Instituto Tecnológico Superior del Occidente del Estado de Hidalgo, Mixquiahuala, Hgo., 42700, Mexico}

\author{Neri Pérez Giovany Humberto}
\affil{Instituto Tecnológico Superior del Occidente del Estado de Hidalgo, Mixquiahuala, Hgo., 42700, Mexico}

%\author{Third Author III}
%\affil{Institute, City, (State), Postal Code, Country}

\markboth{ITSOEH/ITICS/PROYECTO INTEGRADOR PRIMER SEMESTRE}{THEME/FEATURE/DEPARTMENT}

\begin{abstract}
Un resumen (abstract) es un párrafo único que resume los aspectos importantes del manuscrito. A menudo indica si el manuscrito es un informe de un trabajo nuevo, una revisión o una descripción general, o una combinación de ambos. No cite referencias en el resumen. Este tipo de documento debe incluir contenido propiedad de los autores; es decir, no debe contener contenido de otras fuentes, ademas la redacción debe  estar dirigida a un tipo de lector técnico general. Este archivo se encuentra disponible en \href{https://github.com/fcuadrosgithub/integrador-primero.git}{https://github.com/fcuadrosgithub/integrador-primero.git}.
\end{abstract}

\maketitle
\chapteri{L}a introducción debe proporcionar información general (incluidas referencias relevantes) y debe indicar el propósito del manuscrito. En esta sección describa de manera clara y precisa el objetivo del proyecto integrador, la metodología que piensa usar y los resultados obtenidos de manera muy general. Dentro de esta sección puede citar trabajos relevantes de otros si lo cree necesario.

Esta sección debe dar un panorama muy general al lector de cual es el problema a resolver, que metodología utilizó para dar solución al problema y cuales fueron los resultados obtenidos. 

La redacción del manuscrito debe ser en tercera persona y queda estrictamente prohibido el uso de palabras coloquiales o Español informal. En lugar de esto utilice un lenguaje formal que el mayor numero de personas pueda entender.

\section{COPYRIGHT Y ACCESO ABIERTO}

Una vez que los autores entreguen este documento para su evaluación también seden los derechos del contenido de este manuscrito a la carrera de Ingeniería en Tecnológicas de la Información y Comunicaciones (ITICs) del Instituto Tecnológico Superior del Occidente del Estado de Hidalgo (ITSOEH). Esto conlleva que la carrera puede usar el contenido de este articulo para efectos de difusión del quehacer de los estudiantes de la carrera o en cualquier otra actividad que la carrera considere pertinente. Cabe mencionar que en ningún momento el orden o los nombres de los autores sera modificado de ninguna manera y siempre se les dará el crédito correspondiente. 
\section{PROBLEMAS}
A continuación se describen los problemas que el equipo deberá resolver.
\begin{enumerate}
\item Dados 2 puntos $A \mbox{ y } B$ con coordenadas $x_{1}, y_{1}$ y $x_{2}, y_{2}$  respectivamente. Regresar la ecuación de la recta y el ángulo interno $\alpha$ que se forma entre el eje horizontal y la recta. 
%Por ejemplo con los puntos $A(2, 1)$ y $B(-3, 2)$ la ecuación debe ser $y = -\frac{1}{5}x + \frac{7}{5}$. 
\item Dada una ecuación cuadratica regresar los valores de las raíces en caso de que estén sobre el conjunto de los números reales, en caso contrario indicar que la solución esta en el conjunto de los números complejos. 
\item Dada una circunferencia con centro en el punto $C$ con coordenadas $(x_{1}, y_{1})$ y radio $r$, evaluar si un punto $T$ con coordenadas $(x_{2}, y_{2})$ esta dentro del area de la circunferencia.
\item Dado un numero decimal entero positivo o negativo regresar su equivalente en binario.
\item Dado un numero binario de $n$ bits regresar su equivalente en decimal.
\item Dada una tabla de verdad de $n$ bits generar la expresión booleana que genere de manera fidedigna las salidas de esta tabla.
\end{enumerate}

\section{Sección Problema 1}
Contenido del primer problema...
\newpage
Contenido del primer problema...
\newpage


\section{Sección Problema 2: Solucion de ecuaciones cuadraticas sobre el conjumto de numeros reales }
%autor del reporte
\author{
\IEEEauthorblockN{
1\textsuperscript{ro} Irving Maldonado Oguin}
\IEEEauthorblockA{\textit{ITICs} \\
\textit{ITSOEH}\\
Hidalgo, Mixquiahuala \\
230110716@itsoeh.edu.mx}

}
\maketitle

\section{problema 2 La formula general}


\maketitle

\begin{abstract}

El reporte analiza la resolución de ecuaciones cuadráticas o de segundo grado por medio de la fórmula general. Explica la fórmula general y los posibles resultados de la misma. Destaca la importancia de la fórmula y los procesos usados para dar resultados a la fórmula general.
\end{abstract}

\section{Introduction}
 La formula general es util para la resolución de ecuaciones cuadraticas, sirve para encontrar las raices de una ecuacion cuadratica. Esta cuenta con tres reusultados llamados termino cuadratico, termino lineal y termino independiente.


\section{Diagrama de flujo}
A continuacion se muestra el diagrama de flujo el cual es la base del programa

\begin{figure}[h!]
    \centering
    \includegraphics[width = 6 cm]{imagenes/Imagen1.jpeg}
    \caption{Gráfica de la ecuación de la recta}
    \label{fig:GraficaEcuacionRecta}
\end{figure}
















\section{Programa en java}
seguidamente se mostrara el programa de java donde se trabajo con la formula general.
\begin{javaCode}
    
import java.util.Scanner;

public class Main {
    public static void main(String[] args) {
    
        // Crear un objeto Scanner para leer la entrada del usuario
        \begin{javaCode}
        Scanner coeficiente = new Scanner(System.in);
        \end{javaCode}
        // Solicitar al usuario que ingrese el valor de A
        \begin{javaCode}
        System.out.println("Estimado usuario, por favor ingrese el valor de A: ");
        double a = coeficiente.nextDouble();
        \end{javaCode}
        // Solicitar al usuario que ingrese el valor de B
        \begin{javaCode}
        System.out.println("Estimado usuario, por favor ingrese el valor de B: ");
        double b = coeficiente.nextDouble();
        \end{javaCode}
        // Solicitar al usuario que ingrese el valor de C
        \begin{javaCode}
        System.out.println("Estimado usuario, por favor ingrese el valor de C: ");
        double c = coeficiente.nextDouble();
        \end{javaCode}
        // Cerrar el objeto Scanner
        \begin{javaCode}
        coeficiente.close();
        \end{javaCode}
        // Calcular el discriminante de la ecuación cuadrática
        \begin{javaCode}
        double discriminante = b * b - 4 * a * c;
        \end{javaCode}
        // Verificar si el discriminante es mayor que 0
        \begin{javaCode}
        if (discriminante > 0) {
            // Calcular las dos soluciones de la ecuación cuadrática
            double x1 = (-b + Math.sqrt(discriminante)) / (2 * a);
            double x2 = (-b - Math.sqrt(discriminante)) / (2 * a);
        \end{javaCode}
            // Mostrar las soluciones
            \begin{javaCode}
            System.out.println("Las soluciones son x1 = " + x1 + " y x2 = " + x2);
        }
        \end{javaCode}
        // Verificar si el discriminante es igual a 0
        \begin{javaCode}
        else if (discriminante == 0) {
        \end{javaCode}
            // Calcular la solución única de la ecuación cuadrática
            \begin{javaCode}
            double x = -b / (2 * a);
            \end{javaCode}
            // Mostrar la solución única
            \begin{javaCode}
            System.out.println("La solución es: " + x);
        }
        \end{javaCode}
        // El discriminante es negativo, lo que significa que no hay soluciones reales
        \begin{javaCode}
        else {
            System.out.println("no cuenta con soluciones reales.");
        }
    }
}


\end{javaCode}






\section{Tabla de corridas}
\begin{center}
\begin{tabular}{|c|c|c|c|c|}
\hline
No. & A & B & C & Resultado \\
\hline
1 & 19 & 17 & 18 & no cuenta con soluciones reales. \\
\hline
2 & 10 & 15 & 5 & La solución es:$x_1 = -0.5$ y $x_2=-1.0$ \\
\hline
3 & 1 & -5 & 6 & La solución es:$x_1 = 3.0$ y $x_2=2.0$ \\
\hline
4 & 2 & 4 & -6 &La solución es:$x_1 = 1.0$ y $x_2=-3.0$ \\
\hline
5 & 3 & -8 & 4 & La solución es:$x_1 = 2.0$ y $x_2=0.66666$ \\
\hline
\end{tabular}
\end{center}




\section{Conclusión}
La formula general es de mucha utilidad al resolver una ecuación cuadrática. El programa da solución a dichas ecuaciones con la formula general, evitando hacer el paso a paso para resolverlo, muestra al usuario una de las tres soluciones dependiendo de la ecuación, estas soluciones son tres si el discrimínate es mayor a 0 la ecuación cuenta con dos soluciones, si es igual a 0, tienen una única solución y si es menor a 0 no tiene solución real.

\newpage
Contenido del segundo problema...
\newpage


\section{Sección Problema 3}
Contenido del tercer problema...
\newpage 
Contenido del tercer problema...
\newpage 


\section{Sección Problema 4}
Contenido del cuarto problema...
\newpage 
Contenido del cuarto problema...
\newpage 


\section{Sección Problema 5}
Contenido del quinto problema...
\newpage 
Contenido del quinto problema...
\newpage 


\section{Sección Problema 6}
Contenido del sexto problema...
\newpage 
Contenido del sexto problema...
\newpage 



\end{document}

