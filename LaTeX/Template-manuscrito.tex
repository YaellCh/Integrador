\documentclass{IEEEcsmag}

\usepackage[colorlinks,urlcolor=blue,linkcolor=blue,citecolor=blue]{hyperref}
\expandafter\def\expandafter\UrlBreaks\expandafter{\UrlBreaks\do\/\do\*\do\-\do\~\do\'\do\"\do\-}
\usepackage{upmath,color}

\usepackage[spanish]{babel}
%\usepackage[latin1]{inputenc}
\usepackage[utf8]{inputenc}  

\jvol{1}
\jnum{1}
\paper{1}
\jmonth{Noviembre}
\jname{ITICs letters}
\jtitle{Proyectos Integradores}
\pubyear{2023}

\newtheorem{theorem}{Theorem}
\newtheorem{lemma}{Lemma}



\setcounter{secnumdepth}{0}

\begin{document}

\sptitle{Proyecto Integrador de Primer Semestre}

\title{Software de resolución de problemas de Ingeniería }

\author{Yael Antonio Chavez Atanacio}
\affil{Instituto Tecnológico Superior del Occidente del Estado de Hidalgo, Mixquiahuala, Hgo., 42700, Mexico}

\author{Alejandro Cruz Martinez}
\affil{Instituto Tecnológico Superior del Occidente del Estado de Hidalgo, Mixquiahuala, Hgo., 42700, Mexico}

\author{Brayan Hernandez Martinez}
\affil{Instituto Tecnológico Superior del Occidente del Estado de Hidalgo, Mixquiahuala, Hgo., 42700, Mexico}

\author{Irving Maldonado Olguin}
\affil{Instituto Tecnológico Superior del Occidente del Estado de Hidalgo, Mixquiahuala, Hgo., 42700, Mexico}

\author{Gracie Hermione Gutierrez Vazquez}
\affil{Instituto Tecnológico Superior del Occidente del Estado de Hidalgo, Mixquiahuala, Hgo., 42700, Mexico}

%\author{Third Author III}
%\affil{Institute, City, (State), Postal Code, Country}

\markboth{ITSOEH/ITICS/PROYECTO INTEGRADOR PRIMER SEMESTRE}{THEME/FEATURE/DEPARTMENT}

\begin{abstract}
Un resumen (abstract) es un párrafo único que resume los aspectos importantes del manuscrito. A menudo indica si el manuscrito es un informe de un trabajo nuevo, una revisión o una descripción general, o una combinación de ambos. No cite referencias en el resumen. Este tipo de documento debe incluir contenido propiedad de los autores; es decir, no debe contener contenido de otras fuentes, ademas la redacción debe  estar dirigida a un tipo de lector técnico general. Este archivo se encuentra disponible en \href{https://github.com/fcuadrosgithub/integrador-primero.git}{https://github.com/fcuadrosgithub/integrador-primero.git}.
\end{abstract}

\maketitle
\chapteri{L}a introducción debe proporcionar información general (incluidas referencias relevantes) y debe indicar el propósito del manuscrito. En esta sección describa de manera clara y precisa el objetivo del proyecto integrador, la metodología que piensa usar y los resultados obtenidos de manera muy general. Dentro de esta sección puede citar trabajos relevantes de otros si lo cree necesario.

Esta sección debe dar un panorama muy general al lector de cual es el problema a resolver, que metodología utilizó para dar solución al problema y cuales fueron los resultados obtenidos. 

La redacción del manuscrito debe ser en tercera persona y queda estrictamente prohibido el uso de palabras coloquiales o Español informal. En lugar de esto utilice un lenguaje formal que el mayor numero de personas pueda entender.

\section{COPYRIGHT Y ACCESO ABIERTO}

Una vez que los autores entreguen este documento para su evaluación también seden los derechos del contenido de este manuscrito a la carrera de Ingeniería en Tecnológicas de la Información y Comunicaciones (ITICs) del Instituto Tecnológico Superior del Occidente del Estado de Hidalgo (ITSOEH). Esto conlleva que la carrera puede usar el contenido de este articulo para efectos de difusión del quehacer de los estudiantes de la carrera o en cualquier otra actividad que la carrera considere pertinente. Cabe mencionar que en ningún momento el orden o los nombres de los autores sera modificado de ninguna manera y siempre se les dará el crédito correspondiente. 
\section{PROBLEMAS}
A continuación se describen los problemas que el equipo deberá resolver.
\begin{enumerate}
\item Dados 2 puntos $A \mbox{ y } B$ con coordenadas $x_{1}, y_{1}$ y $x_{2}, y_{2}$  respectivamente. Regresar la ecuación de la recta y el ángulo interno $\alpha$ que se forma entre el eje horizontal y la recta. 
%Por ejemplo con los puntos $A(2, 1)$ y $B(-3, 2)$ la ecuación debe ser $y = -\frac{1}{5}x + \frac{7}{5}$. 
\item Dada una ecuación cuadratica regresar los valores de las raíces en caso de que estén sobre el conjunto de los números reales, en caso contrario indicar que la solución esta en el conjunto de los números complejos. 
\item Dada una circunferencia con centro en el punto $C$ con coordenadas $(x_{1}, y_{1})$ y radio $r$, evaluar si un punto $T$ con coordenadas $(x_{2}, y_{2})$ esta dentro del area de la circunferencia.
\item Dado un numero decimal entero positivo o negativo regresar su equivalente en binario.
\item Dado un numero binario de $n$ bits regresar su equivalente en decimal.
\item Dada una tabla de verdad de $n$ bits generar la expresión booleana que genere de manera fidedigna las salidas de esta tabla.
\end{enumerate}

\section{Sección Problema 1}
Contenido del primer problema para Yael...
\newpage
Contenido del primer problema...
\newpage


\section{Sección Problema 2}
Contenido del segundo problema para Irving...
\newpage
Contenido del segundo problema...
\newpage


\subtitle {Problema 3:Cálculo de la distancia entre dos puntos en el área de una circunferencia}
Contenido del tercer problema para Alejandro...
\newpage 
\begin{abstract}
    
    El reporte del problema indica como calcular la distancia de dos puntos y su ubicación ya que se puede encontrar dentro o fuera de una circunferencia además nos da una posible solución a través de un programa en Netbeans
\end{abstract}





\section{Introducción}
El cálculo de la distancia entre dos puntos y determinar si se encuentran dentro de una circunferencia es una tarea común en programación o telecumunicione para ver si una antena de resepcion de red se encuentra en su area .Para esto podemos utilizar un programa en java que realice el calculo de la distancia y su ubicación dado por dos puntos ingresados por el usuario 
\section{ecuación de la distancia y su ubicacion}
En la ecuación de la distancia entre dos puntos $p_1 = (x_1, y_1)$ y $p_2 = (x_2, y_2)$, la distancia entre los dos puntos se calcula utilizando la fórmula de la distancia:
\text{distancia} = \sqrt{{(x_2 - x_1)^2 + (y_2 - y_1)^2}}, para saber si el punto se encuentra dentro del area utilizamos la siguiente formula
\text{ubicacion} = \sqrt{distancia<=radio},si esto se cumple entonces si se encuentra dentro de la circunferencia



\section{Diagrama de flujo}
A continuación veremos el diagrama de flujo que fue la base para el desarrollo del programa
\begin{figure}[h!]
    \centering
    \includegraphics[width = 9 cm]{latex-imagenes/imagen.jpeg}
    \caption{se muestra el diagrama que fue base para diseñar el programa}
    \label{fig:diagrama de flujo}
\end{figure}







\section{Programa en Java}
A continuación se muestra el código en Java para calcular la distancia entre dos puntos y verificar si se encuentran dentro de una circunferencia:


   \begin{javaCode}
       
   

   import java.util.Scanner;
/**
 *
 * @author cruzm
 */
public class cordenadas {

    /**
     * @param args the command line arguments
     */
    public static void main(String[] args) {
        
        
         Scanner dato = new Scanner(System.in);
        \end{javaCode}
        solicitamos al usuario las cordenadas el punto C (x1,y1)
       \begin{javaCode}
        System.out.print("Ingrese las coordenada del centro de la circunferencia primero x1: ");
        int x1 = dato.nextInt();
        System.out.print("Ingrese las coordenada del centro de la circunferencia despues y1: ");
        int y1 = dato.nextInt();
         \end{javaCode}
         solicitamos el radio de la circunferencia
       
        \begin{javaCode}
        System.out.println("Ingrese el radio de la circunferencia: ");
        int radio = dato.nextInt();
         \end{javaCode}
         
        solicitamos al usuario las coordenadas del punto T (x2,y2)
       \begin{javaCode}
        System.out.print("Ingrese las coordenada del punto T X2: ");
        int x2 = dato.nextInt();
        System.out.print("Ingrese las coordenada del punto T Y2: ");
        int y2 = dato.nextInt();
        \end{javaCode}
         Cálculo de la distancia entre el punto C y el punto T
        \begin{javaCode}
        double distancia = Math.sqrt((x2 - x1) * (x2 - x1) + (y2 - y1) * (y2 - y1));
        \end{javaCode}
         imprimimos la distancia
          \begin{javaCode}
         System.out.println("la distancia es de "+distancia);
          \end{javaCode}
       
        si el radio es negativo lo pasamos a positivo
         \begin{javaCode}
        int radio2=radio*-1;
\end{javaCode}
         Si la distancia es menor o igual al radio, el punto T está dentro de la circunferencia     
       
        \begin{javaCode}
                    if(radio<0){
            
        
        int radio2=radio*-1;
        if (distancia <= radio2) {
            System.out.println("El punto T está dentro de la circunferencia");
        } else {
            System.out.println("El punto T no está dentro de la circunferencia");
        }
        }
        
       if(radio>0){
           
       
            if (distancia <= radio) {
            System.out.println("El punto T está dentro de la circunferencia");
        } else {
            System.out.println("El punto T no está dentro de la circunferencia");
        }
    }
    
    
    }
    
}



  
 
    \end{javaCode}

    
   










        \section{tabla de corridas}
 A continuación se muestra una tabla de corridas o pruebas que se realizaron con el programa en java:



    \begin{tabular}{|c|c|c|c|c|c|c|}
\hline
No.corrida & x1 de c & y1 de c & radio & x2 de t & y2 de t & Resultado \\
\hline
No.1 & 1 & 2 & 5 & 2 & 1 & El punto T está dentro de la circunferencia \\
\hline
No.2 & 3 & 1 & 7 & 4 & 5 & El punto T está dentro de la circunferencia \\
\hline
No.3 & 7 & 6 & 3 & 5 & 10 &  El punto T no está dentro de la circunferencia \\
\hline
No.4 & 4 & 5 & 2 & 6 & 8 &  El punto T no está dentro de la circunferencia \\
\hline
No.5 & 2 & 3 & 3 & 4 & 6 &  El punto T no está dentro de la circunferencia \\
\hline
    \end{tabular} 




\section{Conclusión}
El programa se encarga de encontrar la distancia entre dos puntos en el área de una circunferencia. El programa usa la formula de la distancia , con ella encuentra la distancia entre dos puntos, luego es importante solicitar al usuario el radio, se transforma en positivo el radio en caso de ser negativo, después si la distancia es menor o igual al  radio, el punto T esta dentro de la circunferencia.
   


























\newpage 


\section{Sección Problema 4}
Contenido del cuarto problema para Yael ...
\newpage 
Contenido del cuarto problema...
\newpage 


\section{Sección Problema 5}
Contenido del quinto problema para Brian...
\newpage 
Contenido del quinto problema...
\newpage 


\section{Sección Problema 6}
Contenido del sexto problema para Gracie...
\newpage 
Contenido del sexto problema...
\newpage 



\end{document}

